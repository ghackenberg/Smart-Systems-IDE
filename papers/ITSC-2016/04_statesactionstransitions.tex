\section{States, actions and transition function}
Subsequently we describe the states, actions and transition function of the underlying optimal control problem as tuple $(S_t, A_t, T(S_t, A_t))$ in terms of their individual decomposition.

\subsection{States}
States $S_t$ are modeled as a four tuple $
	S_t = (VS_t, ESS_t, CSS_t, RS_t)
$, consisted of vehicle states $VS_t$, energy storage states $ESS_t$, charging station states $CSS_t$ as well as region states $RS_t$.

\subsubsection{Vehicles}
The vehicle state $
	VS_t = (VP_t, VSOC_t)
$ is represented as a tuple, where
\begin{itemize}
	\item[-] $VP_t: VL \rightarrow RSP$ represents a mapping from vehicle labels to region power,
	\item[-] $VSOC_t: VL \rightarrow \mathbb{R}_0^+$ represents a mapping from vehicle labels to their state of charge (i.e. the amount of currently stored energy).
\end{itemize}

\subsubsection{Energy storages}
Energy storage state
$
	ESS_t = (ESOC_t)
$ is represented as a single, where

$ESOC_t: ESL \rightarrow \mathbb{R}_0^+$ represents a mapping from energy storage labels to their current state of charge. 

\subsubsection{Charging stations}
Charging stations state
$
	CSS_t = (CSB_t)
$ is represented as a single, where

$
	CSB_t: CSL \rightarrow \mathbb{R}
$ describes a mapping from charging station labels to the current charging station balance (i.e. the amount of energy sent or received from a connected vehicle).

\subsubsection{Regions}
Region state
$
	RS_t = (RB_t)
$
is represented as a single, where

$
	RB_t: RL \rightarrow \mathbb{R}
$ describes a mapping from region labels to the current region balance (i.e. the aggregated loads of connected electric devices).

\subsection{Actions}
Actions $A_t$ are modeled as a tuple 
$
	A_t = (VA_t, ESA_t)
$, consisted of vehicle actions $VA_t$ as well as energy storage actions $ESA_t$.

\subsubsection{Vehicles}
Vehicle actions
$
	VA_t = (VR_t, VS_t, VB_t)
$ are represented as a three tuple, where
\begin{itemize}
	\item[-]
$
	VR_t: VL \rightarrow (\mathbb{N} \rightarrow RSL)
$ represents a mapping from vehicle labels to a finite number of road segment labels (i.e. route which is followed by a vehicle),
	\item[-]
$
	VS_t: VL \rightarrow \mathbb{R}_0^+
$ describes a mapping from vehicle labels to the current vehicle speeds (i.e. the selected speed the vehicle travels),
	\item[-]
$
	VB_t: VL \rightarrow \mathbb{R}
$ represents a mapping from vehicle labels to vehicle balances (i.e. the amount of energy the vehicle sends or receives from a charging station).
\end{itemize}

\subsubsection{Energy storages}
Energy storage actions 
$
	ESA_t = (ESB_t)
$ are represented as a single, where

$
	ESB_t: ESL \rightarrow \mathbb{R}
$ defines a mapping from energy storage labels to energy storage balances (i.e. the amount of energy which is released from or stored in energy storages).

\subsection{Transition function}
The transition function is a mapping
$
	T: \mathbb{S} \times \mathbb{A} \rightarrow \mathbb{S}
$ describing the transition from a state, given an action to a new state.
Stepping forward in time, the transition function is then defined as a function
\[
	T(S_t, A_t) = S_{t+1} \mathrm{,}
\]where $S_{t+1}$ describes the state following $S_t$ after choosing action $A_t$. Accordingly, the transition function $T$ is then

\[
	T((VS_t, RS_t, ESS_t, CSS_t), (VA_t, ESA_t)) = 
\]
\[
	(VS_{t+1}, RS_{t+1}, ESS_{t+1}, CSS_{t+1}) \mathrm{,}
\]
and contains individual states $S_t$ and actions $A_t$.

The individual vehicle state
\[
	VS_{t+1} = f(VS_t, VA_t)
\]

\[
	ESS_{t+1} = g(ESS_t, ESA_t)	
\]

\[
	CSS_{t+1} = h(VS_{t+1}, VA_t)
\]

\[
	RS_{t+1} = i(CSS_{t+1}, ESA_t)
\]

\subsubsection{Vehicles}

\[
	(VP_{t+1}, VSOC_{t+1}) =
\]
\[
	f((VP_t, VSOC_t), (VR_t, VS_t, VB_t))
\]

\[
	VP_{t+1} = f_1(VP_t, VR_t, VS_t)
\]

\[
	VSOC_{t+1} = f_2(VP_t, VSOC_t, VR_t, VS_t, VB_t)
\]

\subsubsection{Energy storages}

TODO

\subsubsection{Charging stations}

TODO

\subsubsection{Regions}

TODO