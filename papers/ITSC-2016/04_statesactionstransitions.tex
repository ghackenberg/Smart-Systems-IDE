\section{A model of discrete-time \textbf{TransP-0} dynamics}
\label{dynamics}

While the previous section was concerned only with the static parameters of integrated transportation and energy system design, this section focuses on dynamic aspects instead. In effect, each system design defines an optimal control problem (or dynamic programming problem)~\cite{Bertsekas1995} over the transportation and energy subsystem dynamics. In the following, we describe the respective state space in Section~\ref{states}, the action space in Section~\ref{actions}, and the transition function in Section~\ref{transitions}. Note that the states, actions, and transitions do not have to be defined by the transportation and power system engineers. Rather, the definitions are the same for all system design expressed in the \textsc{TransP-0} abstraction.

\subsection{States}
\label{states}

The overall system states $S_t \in \mathbb{S}$ with time point $t \in \mathbb{N}$ of the optimal control problem are modeled as a four-tuple $(VS_t, ESS_t, CSS_t, RS_t)$, where
\begin{itemize}
	\item $VS_t$ represents the states of the \textit{vehicles} introduced in Section~\ref{vehicles},
	\item $ESS_t$ represents the states of the \textit{energy storages} introduced in Section~\ref{energy_storages},
	\item $CSS_t$ represents the states of the \textit{charging stations} introduced in Section~\ref{charging_stations}, and
	\item $RS_t$ represents the states of the \textit{regions} introduced in Section~\ref{regions}.
\end{itemize}
Note that we do not associate a state with the infrastructure of the transportation subsystem (i.e.\ we assume the infrastructure to be constant). In the following, we describe the vehicle states in Section~\ref{states_vehicles}, the energy storage states in Section~\ref{states_storages}, the charging station station states in Section~\ref{states_stations}, and the region states in Section~\ref{states_regions}.

\subsubsection{Vehicles}
\label{states_vehicles}

The vehicle states $VS_t$ of the system state $S_t$ are modeled as a tuple $(VP_t, VSOC_t)$, where
\begin{itemize}
	\item $VP_t: VL \rightarrow RSP$ represents a mapping from vehicle labels to road segment \textit{positions} (see Section~\ref{segments}) and
	\item $VSOC_t: VL \rightarrow \mathbb{R}_0^+$ represents a mapping from vehicle labels to their \textit{state of charge} (i.e. the amount of currently stored energy).
\end{itemize}
Consequently, our design abstraction neglects effects such as changing vehicle weights~\cite{?} due to passenger load or changing friction coefficients due to wheel temperatures~\cite{?}. We omitted such effects to ease design formulation primarily and replaced them with mechanical efficiency coefficients $VME$ (see Section~\ref{vehicles}), which have to be selected carefully to achieve desired statistical effects.

\subsubsection{Energy storages}
\label{states_storages}

The energy storage states $ESS_t$ of the system state $S_t$ are modeled as a one-tuple $(ESOC_t)$, where
\begin{itemize}
	\item $ESOC_t: ESL \rightarrow \mathbb{R}_0^+$ represents a mapping from energy storage labels to their current \textit{state of charge}. 
\end{itemize}
Note that we omitted advanced effects such as wear of equipment, which can cause degrading storage efficiency~\cite{?}. Again, we believe that such effects can be neglected during early phase system-level design. Furthermore, depending on the time step resolution additional state parameters are required to model - for example - ramp-up times of pumped storage hydro power plants~\cite{Garcia2008}.

\subsubsection{Charging stations}
\label{states_stations}

The charging stations states $CSS_t$ of the system state $S_t$ are modeled as a one-tuple $(CSB_t)$, where
\begin{itemize}
	\item $CSB_t: CSL \rightarrow \mathbb{R}$ represents a mapping from charging station labels to the current charging station \textit{balance} (i.e. the amount of energy sent or received from a connected vehicle).
\end{itemize}
In an advanced version of the design abstraction one could also consider failure states or software control states of charging stations. For now we assume that all charging stations work properly. Furthermore, the control strategy is provided implicitly by the optimal control problem formulation.

\subsubsection{Regions}
\label{states_regions}

The region states $R_t$ of the system state $S_t$ are modeled as a one-tuple $(RB_t)$, where
\begin{itemize}
	\item $RB_t: RL \rightarrow \mathbb{R}$ represents a mapping from region labels to the current region \textit{balance} (i.e. the aggregated loads of connected energy subsystem regions and components).
\end{itemize}
Again, we neglect physical state parameters such as power line temperatures or failure modes (e.g.\ due to exceeded temperature limits or due to environmental influences). Consequently, we assume that the energy subsystem infrastructure is available during system operation. In an advanced version of the design abstraction one might also consider failure modes and respective repair actions~\cite{?}.

\subsection{Actions}
\label{actions}

The actions $A_t \in \mathbb{A}$ with time point $t \in \mathbb{N}$ of the optimal control problem are modeled as a tuple $(VA_t, ESA_t)$, where
\begin{itemize}
	\item $VA_t$ represents the actions of the \textit{vehicles} introduced in Section~\ref{vehicles} and
	\item $ESA_t$ represents the actions of the \textit{energy storages} introduced in Section~\ref{energy_storages}.
\end{itemize}
Note that the vehicles and the energy storages are the only system components comprising actions. The states of the other components is influenced directly or indirectly by these actions. In the following, we describe the vehicle actions in Section~\ref{actions_vehicles} before explaining the energy storage actions in Section~\ref{actions_storages}.

\subsubsection{Vehicles}
\label{actions_vehicles}

The vehicle actions $VA_t$ of the system action $A_t$ are modeled as a three-tuple $(VR_t, VS_t, VB_t)$, where
\begin{itemize}
	\item $VR_t: VL \rightarrow (\mathbb{N} \rightarrow RSL)$ represents a mapping from vehicle labels to their respective \textit{route}, i.e.\ a sequence of connected road segments with $\forall vl \in VL, n \in \mathbb{N}:$
	\[
		RST(VR_t(vl)(n)) = RSS(VR_t(vl)(n + 1))
	\]
	starting at the road segment position of the previous vehicle states with $\forall vl \in VL$ and $VP_{t-1}(vl) = (rsl, d):$
	\[
		VR_t(vl)(0) = rsl \textrm{,}
	\]
	\item $VS_t: VL \rightarrow \mathbb{R}_0^+$ represents a mapping from vehicle labels to the current vehicle \textit{speed} (i.e.\ the velocity of the vehicle along the road segments), and
	\item $VB_t: VL \rightarrow \mathbb{R}$ represents a mapping from vehicle labels to vehicle \textit{balances} (i.e.\ the amount of energy sent to or received from a charging station) such that
	\[
		\forall vl \in VL: VB_t(vl) \neq 0 \textrm{ only if }
	\]
	the current vehicle speed is zero (i.e.\ the vehicle does not move) or
	\[
		 VS_t(vl) = 0 \textrm{ and }
	\]
	the vehicle is parked currently at a charging station and connected to the energy subsystem, i.e.\
	\[
		\exists csl \in CSL: VP_t(vl) = (CSP(csl), 0) \textrm{.}
	\]
\end{itemize}
Note that the routes $VR_t$ have to cover the distances traveled by each vehicle with the respective vehicle speeds $VS_t$. Hereby, the vehicle speed also can be zero such that the route only contains the previous road segment. In particular, zero speed is required to park vehicles at charging stations for one time step. Consequently, the time step resolution also determines the time intervals for charging or discharging vehicle batteries. Furthermore, note that we neglect accelerations and decelerations in the model, which might have a considerable effect on energy consumption~\cite{?}. Instead, we assume ideal conditions, which we believe to be sufficient for early phase design.

\subsubsection{Energy storages}
\label{actions_storages}

The energy storage actions $ESA_t$ of the system action $A_t$ are modeled as a one-tuple $(ESB_t)$, where
\begin{itemize}
	\item $ESB_t: ESL \rightarrow \mathbb{R}$ represents a mapping from energy storage labels to energy storage \textit{balances} (i.e.\ the amount of energy sent to or received from the parent region).
\end{itemize}
Note that physically one cannot control the (positive or negative) energy balance directly. Rather, for a pumped storage hydro power plant one might control a valve limiting the downhill water flow and an electric drive causing the uphill water flow~\cite{Castronuovo2004}. In fact, the actual control parameters depend on the concrete storage type. We believe that considering the energy balance directly represents the smallest common denominator.

\subsection{Transition function}
\label{transitions}

The transition function is a mapping
$
	T: \mathbb{S} \times \mathbb{A} \rightarrow \mathbb{S}
$ describing the transition from a state, given an action to a new state.
Stepping forward in time, the transition function is then defined as a function
\[
	T(S_t, A_t) = S_{t+1} \mathrm{,}
\]
where $S_{t+1}$ describes the state following $S_t$ after choosing action $A_t$. Accordingly, the transition function $T$ is then

The individual vehicle state
\begin{itemize}
	\item $VS_{t+1} = f(VS_t, VA_t)$ represents ...,
	\item $ESS_{t+1} = g(ESS_t, ESA_t)$ represents ...,
	\item $CSS_{t+1} = h(VS_{t+1}, VA_t)$ represents ..., and
	\item $RS_{t+1} = i(CSS_{t+1}, ESA_t)$ represents ....
\end{itemize}
\todo{Outline.}

\subsubsection{Vehicles}
\label{transitions_vehicles}

\todo{Introduction.}
\begin{itemize}
	\item $VP_{t+1} = f_1(VP_t, VR_t, VS_t)$ represent ... and
	\item $VSOC_{t+1} = f_2(VP_t, VSOC_t, VR_t, VS_t, VB_t)$ represents ....
\end{itemize}
\todo{Wrap-up.}

\subsubsection{Energy storages}
\label{transitions_storages}

\todo{Formalize!}

\subsubsection{Charging stations}
\label{transitions_stations}

\todo{Formalize!}

\subsubsection{Regions}
\label{transitions_regions}

\todo{Formalize!}