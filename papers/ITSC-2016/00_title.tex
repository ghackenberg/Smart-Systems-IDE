\title{The \textsc{TransP-0} framework for integrated transportation and power system design}

\author{
	\IEEEauthorblockN{Dominik Ascher}
	\IEEEauthorblockA{
		Fakult\"at f\"ur Informatik\\
		Technische Universit\"at M\"unchen\\
		85748 Garching bei M\"unchen, Germany\\
		Email: \href{mailto:ascher@in.tum.de}{ascher@in.tum.de}
	}
	\and
	\IEEEauthorblockN{Georg Hackenberg}
	\IEEEauthorblockA{
		Fakult\"at f\"ur Informatik\\
		Technische Universit\"at M\"unchen\\
		85748 Garching bei M\"unchen, Germany\\
		Email: \href{mailto:hackenbe@in.tum.de}{hackenbe@in.tum.de}
	}
}

\maketitle

\begin{abstract}
	High penetration of electric vehicles (EV) and renewable energy sources (RES) will require fundamental changes to prevalent transportation and power systems. Intermittent and decentralized loads within the power system and the propagation of new transportation paradigms such as the transportation electrification and mobility-on-demand services will impose critical, closely interrelated changes on these systems.
	To guarantee sustainability and minimize negative environmental impacts, closer integration between transportation and power systems is necessary and integrated planing, operation and control strategies for these systems have to be established. In this paper, we present TRANSP-0, a system design abstraction for integrated transportation and power system design. Firstly, we present the components of the design abstraction.  Then we discuss the design abstraction with regard to rapid and iterative system design formulation and evaluation. Finally, we conclude with an outlook on the future scope of our proposed system design abstraction.
\end{abstract}
