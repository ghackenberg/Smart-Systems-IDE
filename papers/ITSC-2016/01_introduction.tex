\section{Introduction and motivation}

Guaranteeing sustainability and minimizing negative environmental impacts are crucial challenges for future power and transportation systems. Widespread adoption of electric vehicles (EVs) as well as high penetration of renewable energy sources (RES) necessitate substantial changes within these systems. Intermittent loads caused by RES require elaborate load balancing strategies within the power system. Increasing adoption of autonomous electric vehicles requires addressing changing mobility demands in the transportation system. To comprehensively address constraints, objectives and design alternatives of both transportation and power systems, sustainable and integrated planning, operation and control strategies have to be established within integrated transportation and power systems.

%To handle strictly volatile energy consumers and producers such as electric vehicles and renewable energy sources, elaborate control strategies or coping mechanisms are required. For instance, distribution grid operators are to decide whether grid expansion provides measurable benefits for it's voltage net users or whether increasing the percentage of smart consumers and producers provides better results.

%More specifically, however, the introduced smartness can possibly result in less predictable parameters for the distribution grid provider to operate on.

According planning, operation and control strategies are frequently addressed within the concept of vehicle-to-grid (V2G). V2G describes a concept, where energy is released from EV batteries to the power system in times of increased power demand.  By facilitating interaction between power system and EVs, V2G possesses the potential to significantly reduce the amount of excess renewable energy produced within the power system \cite{richardson2013electric}. More generally, significant advantages can be argued for widespread V2G adoption, where both environmental and economic benefits have been shown for V2G adoption in the past \cite{richardson2013electric, faria2012sustainability, mwasilu2014electric}. Furthermore, key benefits of V2G include reduction of emissions, increased efficiency as well as stability and reliability of the power system \cite{yilmaz2013review}.

On the contrary, Mwasilu et al.~\cite{mwasilu2014electric} argue that for V2G adoption, central technological issues have to be addressed first such as coping with communication delays, establishing routing protocols and cyber security. Adding to that, the authors argue that frequent charging and discharging cycles causing rapid battery degradation for EV batteries as well as low penetration of electric vehicles with V2G functionality hinder more widespread V2G adoption.

Nevertheless, previous work on plug-in electric vehicles (PEVs) has shown their ability to contribute to balancing the fluctuation of intermittent renewable energy sources \cite{dallinger2012grid}. Currently, V2G represents a supplementary strategy to address situations in the power system with increased power demand. Hence, available intermittent renewable energy sources have to be supplemented with conventional energy sources as well as energy storages within the power system. To achieve a continuous balance between energy supply and demands within power systems, prevalent control schemes rely on automatic control schemes which are supervised by human operators. Here, according control schemes are subject to challenges from increasing levels of fluctuating RES \cite{heussen2012unified}.

Given these current and future challenges and the prospected impact of RES and EVs on such systems, effective integrated control strategies to handle future power and mobility demands still have a long way to go and are subject to ongoing research. Particular demands are constituted by the individual components of power and transportation systems which are subject to a diverse number of objectives and constraints. Hence, to satisfy demands within integrated transportation and power systems and guarantee sustainability elaborate multi-objective optimal control strategies for according integrated systems have to be established.

\subsubsection*{Outline}

The remainder of the article is structured as follows: Section~\ref{related_work} summarizes related work in the field as well as the contributions of this article. Section~\ref{proposed_model} describes the proposed modeling technique. Based on the proposed modeling technique and introduced model. Finally, Section~\ref{conclusion} draws a conclusion from our current state of work.