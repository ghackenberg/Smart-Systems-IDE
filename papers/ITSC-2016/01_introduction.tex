\section{Motivation and differentiation}

%Guaranteeing sustainability and minimizing negative environmental impacts are crucial challenges for transportation and power systems. Widespread adoption of electric vehicles (EVs) as well as high penetration of renewable energy sources (RES) necessitate substantial changes within these systems. 


High numbers of electric vehicles (EV) and renewable energy sources (RES) will require fundamental changes to current transportation and power systems. 
%Intermittent and decentralized loads within the power system and the propagation of new transportation paradigms such as the transportation electrification and mobility-on-demand services will impose interrelated key changes on these systems.
For the power system, uncontrolled charging of a high number of EVs can impose increased peak loads within the distribution network \cite{lopes2009identifying}, while increasing levels of fluctuating RES loads will impose challenges to power system operation \cite{heussen2012unified}. 
However, previous research has shown the ability of plug-in electric vehicles (PEVs) to contribute to balancing the load fluctuations of intermittent RES \cite{dallinger2012grid}. 
Allan et al.~\cite{allan2015benchmark} consider a crucial challenge for successful EV adaption to consist in their integration with supporting infrastructure, i.e. the transportation system, the electric power grid and supporting information systems constituting the intelligent transportation system (ITS). 
Hence, to diminish negative environmental impacts and achieve longterm sustainability, close integration between transportation and power systems is necessary.
%To balance EV and RES behavior, it is a major challenge for transportation systems to guarantee balanced driving behavior.
%Intermittent loads caused by RES require elaborate load balancing strategies within the power system. Increasing adoption of EV requires addressing changing mobility demands in the transportation system. 

%To comprehensively address constraints, objectives and design alternatives of both transportation and power systems, 
To comprehensively address the demands of integrated transportation and power systems, integrated planning, operation and control strategies have to be established. According strategies are frequently addressed within the concept of \textit{vehicle-to-grid (V2G)} \cite{lund2008integration}, which enables energy release from EV batteries to the power system during times of increased power demand. By facilitating interaction between the power system and EVs, widespread V2G adoption provides mutual benefits for transport and energy sectors \cite{lund2006integrated}, possesses the potential to significantly reduce the amount of excess energy produced within the power system \cite{richardson2013electric}, and facilitates both environmental and economic benefits \cite{faria2012sustainability, mwasilu2014electric}. More specifically, key benefits of V2G include reduction of emissions as well as increased efficiency, stability and reliability of the power system \cite{yilmaz2013review}.
On the contrary, Mwasilu et al.~\cite{mwasilu2014electric} argue that for V2G adoption, central technological issues 
%such as communication delays, establishing routing protocols and cyber security 
have to be addressed first, which include rapid battery degradation of EV batteries and the lack of reliable communication protocols.

%and currently low penetration of EVs with V2G functionality.

%To handle strictly volatile energy consumers and producers such as electric vehicles and renewable energy sources, elaborate control strategies or coping mechanisms are required. 
%For instance, distribution grid operators are to decide whether grid expansion provides measurable benefits for it's voltage net users or whether increasing the percentage of smart consumers and producers provides better results.

%Given these current and future challenges and the prospected impact of RES and EVs on such systems, effective integrated control strategies to handle future power and mobility demands still have a long way to go and are subject to ongoing research. 

%Particular demands are constituted by the individual components of power and transportation systems which are subject to a diverse number of objectives and constraints. Hence, to satisfy demands within integrated transportation and power systems and guarantee sustainability elaborate multi-objective optimal control strategies for according integrated systems have to be established.

%\subsubsection*{Outline}
%
%The remainder of the article is structured as follows: Section~\ref{related_work} summarizes related work in the field as well as the contributions of this article. Section~\ref{proposed_model} describes the proposed modeling technique. Based on the proposed modeling technique and introduced model. Finally, Section~\ref{conclusion} draws a conclusion from our current state of work.