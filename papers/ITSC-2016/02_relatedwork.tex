%\section{Differentiation from related work}
%\label{related_work}
%
%In the following we first review related approaches in Section~\ref{approaches} before deriving remaining issues in Section~\ref{problems}. Then, we describe the authors' background in Section~\ref{backgrounds} before summarizing the claimed contributions in Section~\ref{contributions}.

%\subsection{Related approaches}
%\label{approaches}

%Farid et al. \cite{amro2016need} motivate the need for holistic integrated assessment methods managing diversity of control solutions against many competing objectives.
%Galus et al. \cite{galus2012integrating} combine power system models, agent based transport simulations and modeling to model transportation and power systems.


In the context of V2G, methods for intelligent \textit{scheduling} of EV charging/discharging are widely discussed as key approaches to integrate EVs into the power grid 
%by minimizing single or multiple objectives within power systems
\cite{yang2015computational}.
% such as minimizing cost (or maximizing welfare), power losses, emissions, power deviations or optimizing battery performance of EVs within power systems \cite{yang2015computational}. 
%Here, power systems are consisted of a number of electric devices such as conventional or renewable energy sources, energy consumers as well as electric infrastructure. 
%Farid et al. \cite{amro2016need} motivate the need for holistic integrated assessment methods managing diversity of control solutions against many competing objectives.
To sufficiently address technical and economic objectives, Andreotti et al.~\cite{andreotti2012review} argue for better suitability of multi-objective optimization methods over single-objective optimization methods for plug-in vehicles operation in terms of model effectiveness.
Zakariazadeh et al.~\cite{zakariazadeh2014multi} propose a multi-objective Pareto-optimal scheduling method for EVs within a smart distribution network addressing economic and environmental objectives as well as technical constraints managing to reduce operational costs and emissions.
To achieve optimal charging decisions, Ota et al.~\cite{ota2012autonomous} propose a decentralized V2G control scheme addressing the intermittency of RES energy production using electric vehicles. 
%However, the authors focus on the effects of an according charging control scheme within an isolated power system only.
%Another highly relevant direction for efficiently integrating electric vehicles into the power grid and reduce negative impacts is are approaches utilizing Vehicle Routing Problems (VRPs). Methods for vehicle routing typically focus on optimizing route selection for single or multiple traffic participants towards single or multiple objectives and a given set of constraints. Addressing objectives of energy-efficiency in terms of routing problems, Eco-Routing approaches target energy-efficient route selection. In contrast, Eco-Driving approaches target energy-efficient intermediate driving behavior ~\cite{ericsson2006optimizing}. 
%Apart from approaches for optimal scheduling of EV charging/discharging,
In contrast, approaches for \textit{vehicle routing} describe optimizing route selection for single or multiple traffic participants. Hence, they focus on control strategies for route selection/driving instead of charging behavior. 
% towards single or multiple objectives and constraints.
For example, Eco-Routing approaches target energy-efficient route selection for a number of traffic participants \cite{ericsson2006optimizing}. 
Felipe et al.~\cite{felipe2014heuristic} propose multiple heuristics for routing EVs, which consider different partial recharge strategies and technologies while traveling along routes. 
Towards integrating both scheduling and routing approaches, Barco et al.~\cite{barco2013optimal} present an approach for minimizing operation cost for battery EV fleets, which achieves optimal routing and charge scheduling performance. Galus et al. \cite{galus2012integrating} combine power system models, agent based transport simulations and modeling for modeling integrated transportation and power systems.

%the approach does not consider microscopic effects on the power system when making routing and charging decisions in EVs.

%\subsection{Remaining issues}
%\label{problems}

\textit{Problems:} While the presented scheduling methods heavily address control strategies for EVs within the power system, they neglect their effects on the transportation system. In contrast, routing approaches heavily address control strategies for optimally routing single or multiple EVs within the transportation system, but do not incorporate a more detailed representation of the power system and it's underlying objectives. However, the importance of integrated planning for integrating electric and transportation sectors \cite{mathiesen2008integrated} has been shown in the past. In summary, we found that current approaches do not sufficiently support planning and control of integrated transportation and power systems, i.e. the formulation and evaluation of future transportation and power system design options and alternatives.

%Here, approaches often restrict the impact of electric vehicles to decisions on charging or discharging their batteries at charging stations. However, in subsequence, individual EV objectives describing routing preferences such as shortest traveling time or energy-efficiency for EVs cannot be sufficiently taken into account. Instead, emphasis is put on the power system side, while the transportation system including traffic participants isn't represented microscopically.

%VRP
%However, approaches does not take the effects of recharging within the power system into account for general cost evaluation. 
%Hence, we found that current approaches do not address sufficiently the objectives and constraints of both transportation and power systems to holistically estimate the behavior within future power and transportation system scenarios. 
%Therefore, exploring multiple objectives while planning integrated transportation and power systems remains a central issue for stakeholders. 
%To enable rapid adjustment of control strategies has to be considered, when considering rapidly changing control parameters.

%Assessing the balance of the interests of transportation systems contra the power system is a challenge, which has to be tackled in the future.

%\subsection{Claimed contributions}
%\label{contributions}
%\subsection{Authors' background}
%\label{backgrounds}

%In \cite{Hackenberg2012} we presented a model of the electric power system suitable for large-scale computation, which divides the power system into regions and subregions.
%Then, in \cite{ascher2014early} we proposed a model which represents multi-objective traffic flows as an optimal control problem and microscopically captures the mobility demands of individual vehicles within transportation systems.

%Finally, in \cite{ascher2015integrated} we presented a systems modeling technique which allows one to microscopically model and express static and dynamic interaction between components of both power systems and transportation systems.


%\subsection{Claimed contributions}
%\label{contributions}

%In this work, we describe an approach to system design for rapidly varying and evaluating parameters, objective and constraint configurations within integrated transportation and power system scenarios. 
%For this, we formally describe our model for integrated transportation and power systems in terms of the microscopic behavior of individual components. 

\textit{Contributions:} To address this situation, in this paper we extend our previous work \cite{Hackenberg2012, ascher2014early, ascher2015integrated} and present a formal system design framework for rapid and iterative formulation and evaluation of design options within integrated transportation and power systems. For this we describe the underlying design space in terms of static parameters for integrated subsystem design in Section \ref{proposed_model}. Then, we describe the dynamic properties of subsystem design by formulating the underlying optimal control problem in Section \ref{dynamics}. Finally, we establish the requirements to integrated control strategies in terms of objectives and constraints of the described optimal control problem in Section \ref{requirements}.

%of the transportation and power system, i.e. individual cars and electric devices.

%Substantially, in our given model, transportation and power systems are subject to a set of different demands imposed on them. 
%Within the transportation system mobility demands are expressed by passengers, who impose (1)~position preferences including origin and destination of travel as well as (2)~time preferences, which include departure and arrival times. Currently, we restrict the considered transportation modes within our model to electric vehicles.
%Within the power system, power demands are expressed by electric devices in terms of electric power loads (or energy flows) within specific times and durations. To satisfy power demands, the power system employs power generators representing different renewable energy sources.
