\section{Discussion}
\label{section:discussion}
In the following we discuss our approach with respect to four questions: How \textit{valid} are the underlying system models and behavior estimation results? How \textit{novel} is the approach compared to existing studies on electro-mobility? How \textit{efficient} is the technique to evaluate different scenarios? And finally how \textit{applicable} is our approach in practice?
\subsubsection*{Validity}

Our employed system modeling technique enables an early estimation of system behavior and utilizes approximate physical state and dynamics. Model state is evaluated in discrete time steps and employs discrete values for state. Model accuracy could be improved by representing state and time continuously as well as utilizing detailed physical models for transportation systems and power systems, i.e. vehicles and electric devices. However, computational complexity of such models during behavior estimation could prove to be an issue as we currently employ discrete steps to limit possible behavior space and computational complexity to acceptable levels.
In the example scenarios, a model of the traffic network is considered, which neglects low-level traffic infrastructure such as traffic lights and turns within roads. To improve it's representativity, traffic networks with a level of detail of comparable real world traffic infrastructures could be employed. Furthermore, while the behavior of individual cars can be observed microscopically in our model, our vehicle model doesn't consider physically accurate measures of acceleration, deceleration, gravitational forces during turns and energy consumption. Instead, it uses approximate values for traveling speed and energy consumption during simulation steps. Additionally, some objectives of interest to the concept of V2G such as minimization of EV battery degradation and optimization of EV charging/discharging decisions based on energy price have been omitted in our model.
It also remains an issue whether configurations employed in our example scenarios are representative for real world scenarios. To obtain a measure of validity, the quality of results obtained with our approach has to be compared with detailed simulations of traffic and electrical networks.

%In terms of the power system, 
%a basic net architecture consisting of two levels of low and medium voltage nets has been considered within the scope of our model, while higher voltage nets have been omitted. T
%the specific structure of voltage nets employed in the presented example scenarios is not representative for real world voltage nets and represents an high-level approximation. 
%Additionally, static load components proposed in our model represent high-level abstractions with diminished model accuracy of the power system.
% and aggregate the behavior, i.e. the power loads of a wide range of electric devices.

\subsubsection*{Novelty}

We employ a parameter-based approach allowing rapid modeling and evaluation of multi-objective transportation and power system scenarios. Compared to existing approaches outlined in Sec.~\ref{section:retrospection}, our approach allows quick and adaptable model formulation with regard to transportation and power systems. Furthermore, our approach supports multiple objectives, which can be rapidly established or adjusted for systems or subsystems.
%Fundamentally, our approach allows one to model situations within transportation as well as power systems and their interaction. 
As demonstrated in Sec.~\ref{section:evaluation}, this enables one to rapidly evaluate scenarios concerning evolving systems, e.g. different expansion levels of renewable and smart energy devices or changing system objective focus.

\subsubsection*{Efficiency}

Our approach represents a rapid prototyping technique enabling to model, estimate and evaluate transportation and power system scenarios. For this, we employ incremental techniques allowing model refinement through multiple iterations. Here, the efficiency of our approach can be measured in terms of lines of code necessary to establish a specific scenario. Reducing time and resource intensive modeling, in the presented example scenarios our approach enabled to quickly establish different scenarios constantly within 350 lines of code without requiring extensive adjustments. 
%Lines of code represent a concise measure for the ease of establishing scenarios with our modeling technique.

\subsubsection*{Applicability}

%Planning transportation and power systems, the applicability of our approach has to be shown. 

The practical applicability of our approach is strongly dependent on employed domain knowledge specific for transportation and power systems. Here, a key issue is the fragmentation of knowledge into different domains and the autonomous goals involved for stakeholders. Therefore, the practical applicability of our modeling technique depends on different domains collaborating and sharing their knowledge.