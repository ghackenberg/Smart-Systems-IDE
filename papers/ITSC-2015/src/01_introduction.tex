\section{Introduction (1.5 pages)}

In recent years, increasing awareness of environmental impacts has sparked increased interest in electric vehicles (EV) as a means to diminish dependence on fossil fuels. EVs can provide a sustainable alternative to conventional fuel vehicles if their specific demands can be sufficiently addressed by electricity networks and transportation systems.
Intermittent impacts on the power grid caused by EVs remain a major challenge that needs to be addressed, as the sustainability of the widespread adaption of EVs is strongly related with efficiently balancing power demands within power grids. 

Additional complexity is caused by the broad range of possible interactions of EVs with the power grid. Vehicle-to-Grid (V2G) is a concept which considers releasing stored electrical energy from EV batteries to the power grid. Interactions arising from this concept are diverse in their nature and intended objectives. For one, when connected to the power grid, the batteries of electric vehicles can supply power during times of increased demand in the grid. Resulting benefits from V2G can be seen for both the electric grid sector and the transportation sector ~\cite{tomic2007using}. While major benefits for the grid may be seen in efficiency, stability and reliability, some of the problems such as battery degradation, need for intensive communication, necessary infrastructure changes, as well as general social, political, cultural, and technical obstacles could present major impediments ~\cite{yilmaz2013review}. Ideally, the locality of power demand of power grids is addressed in charging and discharging decisions based on the impact on the power network. Impacts on the power grid of vehicle carried batteries have been studied in the past to have significant economic impact ~\cite{peterson2010economics,erdinc2014economic}. In practice, however, it has been shown that the majority of EV Users (80\%) are favorable of charging their vehicles at home ~\cite{haines2009simulation}, which may present a customer acceptance problem of V2G. More importantly, this manifests the strong difference between achieving objectives involved in charging decisions from the perspective of the power network on the one side and objectives of transportation systems with individual electric vehicles on the other side. Therefore, for engineering future transportation systems and electric networks it is important to understand the specific objectives and how they interact.

In the past, research in the context of EV charging scheduling has strongly focused on achieving optimal charging decisions for EVs with regard to the power network. Sundstrom et al. \cite{sundstrom2010planning} consider the capacity of the electricity grid during charging of electric vehicles, thereby avoiding overloading in the grid. Schlote et al. \cite{schlote2012balanced} implement load-balancing strategies for charging demand at charging points into the routing decisions of electric vehicles. Results showed reduced travel times and congestion within the employed traffic network. Deliami et. al \cite{deilami2011real} propose a real-time smart load management, which considers random plug-in of plug-in electric vehicles (PEVs) within smart grids, considering scenarios with intermittent charging of PEVs. Alonso et al. \cite{alonso2014optimal} examine optimal charging scheduling for electric vehicles in smart grids, achieving optimal behavior for charging EVs in a power grid based on a representative low-voltage net topology. The objective for the employed model lies in obtaining daily optimal scheduling of EV demand on transformer substations. Zakariazadeh et. al \cite{zakariazadeh2014multi} propose a multi-objective scheduling approach for electric vehicles in smart distribution systems, which considers economic and environmental objectives as well as technical constraints of distribution networks. In summary, for representation of a underlying power system, proposed approaches do not model electric devices in detail or only rely on representative electric net topologies or total charging demand. Furthermore, a problem of the presented approaches represents the missing capability to explore holistic scenarios for the transportation and the power system. Considered approaches foremost aim to address the objectives and constraints of the power system, while objectives and constraints of the transportation system are not sufficiently considered.

Traditionally, SUMO \cite{behrisch2011sumo} has seen widespread attention for microscopic simulation of traffic and it's straightforward formulation as well as evaluation of large-scale traffic scenarios. In the context of EVs, an extension of SUMO towards modeling electrical and mechanical traction of EVs has been proposed in \cite{maia2011electric}. Kurcveil et. al \cite{kurczveil2014implementation} propose an implementation of an energy model and a charging infrastructure in SUMO. The approach allows for modeling representative EV traffic scenarios in SUMO. Soares et al. \cite{soares2012electric} propose an electric vehicle scenario simulator which enables the definition of electric vehicle scenarios in the context of smart grids and distribution networks. However, it's scope for is limited to the definition of electric vehicles scenarios and relies on external tools for specific analysis and determining impact on the power grid. Furthermore, the height profile, i.e. the elevation slope of routes is not considered in their energetic profile. An addition to this approach has been proposed in a methodology for generating realistic traffic scenarios based on synthetic population generation utilizing a census database \cite{soares2014realistic}.
In summary, the presented approaches propose possibilities for parameter variation and modeling of traffic scenarios for EVs. However, for one they microscopically consider the transportation system with it's EVs, but lack possibilities to explore the behavior of the underlying power system or specific electric devices in detail. Consequently, in scenarios, they do not offer possibilities for detailed parameter variation of the power system and it's individual devices. Therefore, holistic transportation scenarios considering the behavior of transportation systems as well as power systems cannot be explored and evaluated.

\subsection{Problem}

After studying related work, we have identified two major problems with existing approaches to future scenario modeling and exploration: First, only very basic models of the electricity network and its components are used such that a number of effects cannot be studied effectively. And second, only limited support is provided for varying the parameters of future scenarios systematically such that important variations might not be considered at all.

\subsection{Contribution}

To overcome the current situation we extend an existing approach to transportation system modeling and scenario exploration such that the electricity network can be represented in greater detail. Then we derive a systematic approach to modeling and varying the parameters of future transportation system and electricity network scenarios. Finally, we demonstrate the approach using a number of examples and discuss its strengths and weaknesses.

\subsection{Outline (Streichkandidat)}

In the following, we describe the underlying modeling technique of our approach in Section~\ref{section:contribution_0} followed by our approach to integrated transportation system and electricity network modeling in Section ~\ref{section:contribution_1}. Thereon we discuss a number of examples and evaluate our approach in Section~\ref{section:evaluation}. Finally we conclude with a summary of our contributions, remaining deficiencies and future work in Section~\ref{section:conclusion}.