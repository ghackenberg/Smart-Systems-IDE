\section{Introduction (1 page)}

{\color{red} Wie soll die Argumentation hier verlaufen?}

Increasing awareness of the environmental impacts of driving has spawned many recent approaches addressing energy-efficient vehicle behavior in the context of Intelligent Transportation Systems (ITS). Taking consistent steps towards CO2 reduction and independence from fossil fuels, in recent time, hybrid electric vehicles (HEV) and electric vehicles (EV) have seen increased attention in research. 
Furthermore, the impact on the power distribution network remains a challenge for electric vehicles with Grid-to-Vehicle (G2V) as well as Vehicle-to-Grid (V2G) interactions to consider.

\subsection{Problem}

After studying related work, we have identified two major problems with existing approaches to future scenario modeling and exploration: First, only very basic models of the electricity network and its components are used such that a number of effects cannot be studied effectively. And second, only limited support is provided for varying the parameters of future scenarios systematically such that important variations might not be considered at all.

\subsection{Contribution}

To overcome the current situation we extend an existing approach to transportation system modeling and scenario exploration such that the electricity network can be represented in greater detail. Then we derive a systematic approach to modeling and varying the parameters of future transportation system and electricity network scenarios. Finally, we demonstrate the approach using a number of examples and discuss its strengths and weaknesses.

\subsection{Outline}

In the following, we first provide an overview of related work in Section~\ref{section:related_work}. Then we describe our approach to integrated transportation system and electricity network modeling in Section~\ref{section:contribution_1} followed by our approach to systematic parameter variation in Section~\ref{section:contribution_2}. Thereon we discuss a number of examples in Section~\ref{section:evaluation}. Finally we conclude with a summary of our contributions, remaining deficiencies and future work in Section~\ref{section:conclusion}.