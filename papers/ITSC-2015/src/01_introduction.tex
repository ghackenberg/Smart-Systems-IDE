\section{Introduction}

In recent years increasing environmental awareness has sparked interest in electric vehicles (EVs) as one possible means to reduce dependence on fossil fuels. However, up until now undesired impacts on the power system remain a major challenge that needs to be addressed before their safe widespread adoption. In this context, inherent complexity is caused by the broad range of constraints, objectives and design alternatives both for the transportation and the power system as well as their interaction. For example, vehicle-to-grid (V2G) is a concept that considers releasing energy from EV batteries to the power system during times of increased power demand. Resulting benefits can be argued both for the transportation and the power system~\cite{tomic2007using}. While major benefits include efficiency, stability and reliability of the power system, major impediments encompass faster battery degradation, need for intensive communication and infrastructure changes as well as general social, political, cultural and technical obstacles~\cite{yilmaz2013review}. Still, possible impacts have been shown to comprise significant economic potential~\cite{peterson2010economics,erdinc2014economic}. Despite technical challenges, practice shows that today the majority of EV users (i.e.\ about 80\%) favor charging their vehicles at home~\cite{haines2009simulation}, which may present an acceptance problem for V2G. More importantly, the V2G example manifests the strong difference between objectives of the power system on the one side and the transportation system on the other side including their respective stakeholders. Therefore, when engineering future transportation and power systems it is important to understand their specific constraints and objectives as well as their interactions and impacts on each other. Moreover, we believe it is beneficial to estimate possible impacts as early as possible such that costly design mistakes can be avoided.

\subsection{Problem}

Our research focuses on methods and tools, which allow one to estimate possible impacts arising from integrating  transportation and power systems as early as possible in the design process. Hereby, we are interested in microscopic effects that can be observed on the individual car and electric device level as well as the distribution network (i.e.\ low and medium voltage network). However, at this stage we consider approximate physical state and dynamics to be sufficient.

\subsection{Contribution}

We extend an existing approach to transportation system modeling and scenario exploration to include a suitable representation of the power system as well as their interaction. Furthermore, we develop an approach to both modeling the parameters of the transportation and the power system, such that different scenarios can be configured rapidly. Finally, we demonstrate the approach using a set of examples and discuss its current strengths and weaknesses.

\subsection{Outline}

First, we summarize related work and remaining deficiencies in Section~\ref{section:retrospection}. Then, we describe the underlying modeling technique of our approach in Section~\ref{section:foundation}. Subsequently, we explain our approach to integrated transportation and power system modeling in Section ~\ref{section:contribution}. Thereon, for demonstration, we elaborate on four scenarios in Section~\ref{section:evaluation}. In Section~\ref{section:discussion} we provide an extensive discussion. Finally, we conclude in Section~\ref{section:conclusion}.