\section{Conclusion}
\label{section:conclusion}

In this paper, we proposed an approach to rapidly model and evaluate transportation and power system scenarios. For this, we extended an existing approach to transportation system modeling and scenario exploration to include a suitable representation of the power system.
The proposed holistic and component-based approach allows to model power systems with respective electric devices and infrastructures as well as transportation systems with individual cars. Our approach allows for rapidly varying the parameters within transportation system and electric network scenarios. For this, we evaluated several incremental traffic scenarios, whose parameters have been varied in terms of stage of expansion of renewable and smart energy devices and in terms of varying the weight of objectives of transportation and power systems. Scenario results showed the feasibility of the approach in terms of rapidly evaluating interacting systems as well as system composition scenarios.
%Findings demonstrated the feasibility of our approach for holistic transportation system and electric network scenario modeling. 
Future work includes refinement of model accuracy through vehicle model improvements and utilization as well as modeling of additional objectives and systems. Furthermore, utilization of OpenStreetMap to model representative traffic infrastructures and validation of results obtained with our approach represent major next steps. Finally, to be able to handle larger problem scales and computational complexity more easily, we currently work on distributed behavior estimation algorithms.

%Integration in SUMO. 
%Independently validation of behavior estimation results could be shown by comparison to SUMO implementation.
%Further extension of our approach could include the modeling and parametrization of additional systems such as weather
%Balancing local voltage net loads 