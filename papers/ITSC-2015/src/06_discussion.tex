\section{Discussion}
\label{section:discussion}

In the following we discuss our approach with respect to four questions: How \textit{valid} are the underlying system models and the behavior estimation results? How \textit{novel} is the approach compared to existing studies on electro-mobility? How \textit{efficient} is the technique to evaluate different scenarios? And finally how \textit{applicable} is our approach in practice?

\subsection{Validity}

%The scope of our approach is to provide an overview of possible behavior during early phases of systems engineering. 
In our approach, the employed system modeling technique considers high-level models, which provide an estimation of system behavior utilizing approximate physical state and dynamics. It has to be shown whether resulting behavior estimation results of approach prove to be valid compared with modeling and simulation approaches providing detailed simulations of traffic and electrical networks. In presented example scenarios, the model is evaluated in discrete time steps of 60 seconds over a total time of 30 minutes, and employs discrete values of state. Model accuracy could be improved by modeling values of state and time in continuous steps. However, computational complexity of such models during behavior estimation could represent an issue as we currently employ discrete steps to limit possible behavior space and computational complexity.

In terms of the traffic system, a representation of a traffic network has been employed, on which behavior of individual cars can be observed microscopically. Currently, a traffic network of 45 kilometers in length and 45 kilometers in width is considered. However, in the presented example scenarios, the model of the traffic network neglects low-level infrastructure such as traffic lights and turns within roads. To improve on the representativity of utilized traffic infrastructures, more detailed traffic networks reflecting real world traffic infrastructures could be employed. Furthermore, in terms of the transportation system, in our model, the vehicle model employed by cars doesn't consider acceleration, deceleration and gravitational forces during turns, but uses randomly selected average values for traveling speed on edges instead. Furthermore, energy consumption is based on a measure of energy efficiency, traveled elevation profile, car weight as well as randomly selected speed. In this context, it also remains an issue, whether parameters employed in our example scenarios are representative for real world scenarios. Additionally, to reduce model complexity, several objectives of EVs which are of main interest to drivers such as minimization of battery degradation and minimization/maximization of energy price during charge/discharge have been neglected in our model, but represent important issues of the V2G concept.

In terms of the power system and electric infrastructure, only a basic net architecture consisting of two levels, i.e. low and medium voltage nets, has been considered in our model, while higher voltage nets are currently omitted. Furthermore, the specific structure of voltage nets employed in example scenarios is not representative for real world voltage nets. Additionally, static load components proposed in our model represent high level abstractions with diminished model accuracy and aggregate the behavior, i.e. the power loads of a wide range of electric devices.

%In the results obtained from the examples, findings showed that net balance equalization was achieved to a higher degree, when the weight assigned to the costs of the power system was higher, in contrast to results achieved when higher weight was assigned to the costs of the transportation system. 
%To reduce model complexity, several objectives of EVs which are of main interest to drivers such as minimization of battery degradation and minimization/maximization of energy price during charge/discharge have been neglected in our model. However, both but are important issues of the V2G concept which have to be addressed in the future.
%Generally, It remains an issue, whether an distinction between objectives of transportation and power systems can be applied to real world scenarios when considering a very high number of potentially interacting objectives. 

\subsection{Novelty}

We employ a parameter-based approach allowing for rapid modeling and evaluation of multi-objective transportation and power system scenarios. Compared to existing approaches outlined in Sec.~\ref{section:retrospection}, model formulation can be achieved in terms of individual composition of involved systems and with regard to multiple objectives. Fundamentally, our approach allows one to model situations within transportation as well as power systems and their interaction. As demonstrated in Sec.~\ref{section:evaluation}, this enables one to rapidly evaluate scenarios concerning different systems, e.g. different expansion levels of renewable and smart energy devices.

\subsection{Efficiency}

At its core, our approach represents a rapid prototyping technique, enabling to model, estimate and evaluate scenarios concerning transportation and power systems. For this, we employ incremental techniques which allow model refinement through multiple iterations. Here, the efficiency of our approach can be measured in terms of lines of code necessary to establish a specific scenario. Avoiding large portions of time and resource intensive modeling, in example scenarios our approach enabled to quickly establish small-scale scenarios within 350 lines of code. However, multiple iterations when establishing scenarios have to be factored in, as lines of code do not necessarily represent a concise measure for effort, but rather for ease of scenario modeling.

\subsection{Applicability}

In terms of planning transportation and power systems of the future, the applicability of our approach remains an open issue. In this context, practical applicability is strongly dependent on domain knowledge specific for transportation and power systems. A major issue represents fragmentation of knowledge into different domains and autonomous goals involved for different stakeholders of transportation systems and electric networks. Therefore, practical applicability of transportation and power system scenario modeling and evaluation depends on domains collaborating and sharing domain knowledge.