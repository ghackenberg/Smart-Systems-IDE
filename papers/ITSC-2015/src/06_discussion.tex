\section{Discussion}
\label{section:discussion}

In the following we discuss our approach with respect to four questions: How \textit{valid} are the underlying system models and the behavior estimation results? How \textit{novel} is the approach compared to existing studies on electro-mobility? How \textit{efficient} is the technique to evaluate different scenarios? And finally how \textit{applicable} is our approach in practice?

\subsection{Validity}

The scope of our approach is to provide an overview of possible behavior during early phases of systems engineering. In this context, for system modeling we employ high-level models, which provide an estimation of system behavior. Therefore, it has to be shown whether according behavior specification and estimation prove to be valid compared with detailed traffic and electrical network simulations. In presented examples, model state is evaluated in steps of 60 seconds over all total time of 30 minutes, showing only a small portion of behavior over time. However, computational complexity of models during behavior optimization could represent an issue, as continuous values of time and state have to be considered. In contrast, in our model, values of state and time are modeled in discrete steps to reduce computational complexity.

The proposed approach provides an estimation over traffic and energy flows within high-level scenarios. In terms of the traffic infrastructure, a representation of the traffic network has been employed, on which behavior of individual cars can be observed microscopically. Currently, a traffic network of 45km in width and 45km in length is considered. More detailed traffic networks could be employed to more accurately represent real world traffic infrastructures. Our current traffic network neglects low-level infrastructure such as traffic lights and turns within roads, i.e. bended roads. In our model, a directed graph consisted of nodes and edges is employed instead, which imposes constraints on traffic flows through node and edge capacities. Furthermore, the currently employed vehicle model doesn't consider acceleration, deceleration and gravitational forces during turns, but uses average values for speed and energy consumption instead. Centrally, it also remains an issue, whether the parameters employed in our scenarios are representative for real-world scenarios.

In terms of the power system and electric infrastructure, only a basic net architecture consisting of two levels, i.e. low and medium voltage nets, is considered, neglecting a representative structure found in real world circumstances. Further abstraction is employed by static profile components, which aggregate intermittent net loads over time, representing high-level abstractions with diminished accuracy. Furthermore, charging and discharging of cars at charging stations utilizes static loads in contrast to variable loads, due to reduction of possible behavior space and computational complexity. In Summary, comparison between our approach and implementation in a simulation framework like SUMO could provide a means for validation of our approach, especially in terms of behavior the transportation system.

In the results obtained from the examples, it has been shown that total load balance equalization was achieved to a higher degree, when the weight of the power system cost function was higher. In contrast, when more weight was put on the transportation system cost function, findings showed total load balance equalization to a lower degree. In this context, it remains an issue, whether such an objective distinction between different systems and according objective parameter variation can be applied to real world scenarios when considering a very high number of potentially involved objectives. Furthermore, to reduce complexity, several objectives of EVs which are of main interest to drivers such as battery degradation and energy price have been neglected in our model, but are important issues of the V2G concept which have to be addressed in the future.

\subsection{Novelty}

We employ a parameter-based approach allowing for rapid modeling and evaluation of multi-objective transportation and power system scenarios. Compared to existing approaches outlined in Sec.~\ref{section:retrospection}, model formulation and adjustment can be rapidly achieved in terms of individual composition of involved systems, independently of underlying objectives. Fundamentally, our approach allows one to concisely assess situations within transportation and power systems. As demonstrated in Sec.~\ref{section:evaluation}, this enables one to rapidly evaluate novel scenarios concerning different systems, e.g. different expansion levels of renewable and smart energy penetration. 

\subsection{Efficiency}

At its core, our approach represents a rapid prototyping technique, enabling to formulate, estimate and evaluate scenarios concerning multiple systems. For this, we employ incremental techniques which allow model refinement through multiple iterations. Here, the efficiency of our approach can be measured in terms of lines of code (LOC) necessary to establish a given scenario. Avoiding time and resource intensive modeling, our approach has shown to allow quickly establishing medium complexity scenarios within 350 LOC. However, multiple iterations when establishing scenarios have to be factored in, as LOC do not represent an accurate measure for effort, but rather for ease of modeling.

\subsection{Applicability}

Planning transportation and power systems, for our approach it remains an issue of applicability allowing to rapidly evaluate scenarios. The applicability of the approach is strongly dependent on involved domain knowledge, providing insight into involved systems. However, a major issue represents the fragmentation into different domains and involved goals with transportation systems being regulated by government bodies and electric networks being regulated by private entities. Therefore, practical applicability of transportation and power system modeling and evaluation depends on separated domains collaborating.