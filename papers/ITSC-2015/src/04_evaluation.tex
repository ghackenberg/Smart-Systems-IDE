\section{Demonstration by examples (1.5 pages)}
\label{section:evaluation}

To demonstrate the presented approach, we show a set of different examples with diverse scenario configurations and employed model parameters. Specifically, the proposed examples evaluate the effects of different weight balances between the objectives of the transportation and the power system. 
Furthermore, in the examples, different levels of smart and renewable energy penetration are considered. 


Among all examples, model state is evaluated every 60 seconds to a total of 30 times. This represents a time resolution of 60 seconds with a total observed duration of 30 minutes. 
In terms of aggregated costs, weights of the individual cost functions of power and transportation systems are alternated in the intelligent transportation and power system.

In terms of the power system and it's electric devices, examples include 10 static load components, 5 solar panels, 5 batteries and depending on example, 16 or 56 charging stations. In terms of electricity infrastructure, 4 low voltage nets and 1 medium voltage net are employed. Between examples 1-2 and examples 3-4 power system parameters are varied in terms of the number of charging stations as well as in terms of solar panel and battery capacities. Specifically, the number of charging stations from examples 1 and 2 is increased to the total number of nodes on the traffic network, i.e. 56, in examples 3 and 4. Additionally, from examples 1-2 to examples 3-4, solar panel capacities are increased twofold, while battery capacities are increased fourfold. 

For the transportation system, examples feature a total number of 240 cars. Cars are divided in equal numbers between reference types $C_{A}$, $C_{B}$ and $C_{C}$ with the difference between reference types respectively being state of charge levels at 33\%, 66\% or 99\% of maximum levels. In terms of positioning on the traffic network, origin position of individual cars is randomly drawn from all edges present in the traffic network. Destination positions are randomly drawn from a set of 8 edges of most outward edges of the traffic network. For both origin and destination selection, we employ uniform probabilistic distribution on available options. 

Specifically, differences between the examples for the transportation system and it's subcomponents are described in Table~\ref{tab:example1}. 

\begin{table}[h]
	\renewcommand{\arraystretch}{1.3}
	\caption{Example Overview}
	\label{tab:example1}
	\centering
	\begin{tabular}{llllll}
		\hline
		\textbf{Parameter/Reference Type}                    & \textbf{Ex. A}    & \textbf{Ex. B} & \textbf{Ex. C} & \textbf{Ex. D}\\ \hline
		Weight Transp. System 			& 0.75	      & 0.25  	& 0.75	& 0.25\\
		Weight Power System 			& 0.25	      & 0.75  	& 0.25	& 0.75\\
		Number Charging Stations              & 16         & 16 		& 56	& 56\\
		Capacity Batteries          & 100 kW/h         & 100 kW/h 		& 400 kW/h		& 400 kW/h\\
		Capacity Solar Panels               & 25 kW/h         & 25 kW/h 		& 50 kW/h		& 50 kW/h	\\ \hline
		Intelligent TS and EN                 & $IS_{A}$         & $IS_{B}$ 		& $IS_{A}$		& $IS_{B}$	\\ 
		Transportation System                 & $TS_{A}$         & $TS_{A}$ 		& $TS_{A}$		& $TS_{A}$	\\ 
		Electric Network                & $EN_{A}$         & $EN_{A}$ 		& $EN_{B}$		& $EN_{B}$	\\ 
		Cars                  & $C_{A,B,C}$          & $C_{A,B,C}$		& $C_{A,B,C}$		& $C_{A,B,C}$	\\ 
		Low Voltage Nets                 & $LV_{A}$         & $LV_{A}$ 		& $LV_{A}$		& $LV_{A}$	\\ 
		Medium Voltage Nets                 & $MV_{A}$         & $MV_{A}$ 		& $MV_{A}$		& $MV_{A}$	\\ 
		Charging Stations                 & $CS_{A}$         & $CS_{A}$ 		& $CS_{A}$		& $CS_{A}$	\\ 
		Power Batteries                & $PB_{A}$         & $PB_{A}$ 		& $PB_{B}$		& $PB_{B}$	\\ 
		Solar Panels                 & $SP_{A}$         & $SP_{A}$ 		& $SP_{B}$		& $SP_{B}$	\\ 
		Static Loads                 & $SL_{A}$         & $SL_{A}$ 		& $SL_{B}$		& $SL_{B}$	\\ \hline
	\end{tabular}
\end{table}

\subsubsection{Example 1}
Example 1 describes a scenario with a low number and low capacities of smart and renewable energy devices. That is, low solar panel capacity, low battery capacity is available and static profile simulating net influence has high impact. Also, only a low number of charging stations is available for cars. More weight is put on the costs incurred by transportation instead of the power system, in benefit of achieving the objectives of the transportation system. Behavior estimation results show low frequency of edges around and on charging stations. Furthermore, net balance is prone to high and sudden fluctuations in load. Little equalization of negative net balances is made by cars discharging at charging stations. In result, incurred transportation system costs are very high, while power system costs are low.

\subsubsection{Example 2}

In Example 2, numbers and capacities of smart and renewable energy devices are equal to Example 1. However, different to Example 1, more weight is put on the costs incurred by power system instead of the transportation system, in benefit of achieving the objectives of the power system. Behavior estimation results show higher frequency of edges around and on charging stations compared to example 1. Also in contrast to Example 1, net negative balances are equalized more heavily, which can be traced to cars discharging at charging stations. While energy capacity of storages and solar panels is low and load balancing only occurs to small degree, balancing is more heavily utilized compared to Example 1. Due to objective weight, costs of the power system are higher than in example 1, but in combination with the transportation system, lower total costs are achieved.

\subsubsection{Example 3}

In Example 3, smart and renewable energy penetration is increased through additional solar panel and battery capacity as well as a higher number of charging stations distributed on the traffic network. Also, the profile of static loads is more evened out, as the power grid and it's electric devices get more smart resulting in a less power load peaks. 
Equal to example 1, more weight is put on the costs incurred by transportation instead of the power system, in benefit of achieving the objectives of the transportation system. Behavior estimation results show low frequency of edges around and on charging stations. Furthermore, net balance is prone to high and sudden fluctuations in load. Equalization of negative net balances is seen by higher capacities of solar panels and batteries as well as a higher number of charging stations, in contrast to Examples 1 and 2. In result, costs are much lower compared to examples 1 and 2.

\subsubsection{Example 4}

In Example 4, numbers and capacities of smart and renewable energy devices are equal to Example 3. Different to example 3, weight of power system costs is greater than weight of transportation system costs. Behavior estimation results show higher frequency of edges around and on charging stations compared to example 3, however, compared to example 2, frequency is distributed more evenly across the traffic network. In contrast to Example 3, negative net balances are equalized more evenly. In result, this leads to very low power system and low transportation system costs.

\section{Discussion (0.75 Pages)}
\label{section:discussion}

\subsection{Validity}

The scope of our approach is to provide an overview of possible behavior during early phases of systems engineering. In this context, for system modeling we employ high-level models, which provide an estimation of system behavior. Therefore, it has to be shown whether according behavior specification and estimation prove to be valid compared with detailed traffic and electrical network simulations. In presented examples, model state is evaluated in steps of 60 seconds over all total time of 30 minutes, showing only a small portion of behavior over time. 
However, computational complexity of models during behavior optimization could represent an issue, as continuous values of time and state have to be considered. In contrast, in our model, values of state and time are modeled in discrete steps to reduce computational complexity. In our model, 

The proposed approach provides an estimation over traffic and energy flows within high-level scenarios. In terms of the traffic infrastructure, a representation of the traffic network has been employed, on which behavior of individual cars can be observed microscopically. Currently, a traffic network of 45km in width and 45km in length is considered. More detailed traffic networks could be employed to more accurately represent real world traffic infrastructures. Our current traffic network neglects low-level infrastructure such as traffic lights and turns within roads, i.e. bended roads. In our model, a directed graph consisted of nodes and edges is employed instead, which imposes constraints on traffic flows through node and edge capacities. Furthermore, the currently employed vehicle model doesn't consider acceleration, deceleration and gravitational forces during turns, but uses average values for speed and energy consumption instead. In terms of the power system and electric infrastructure, only a basic net architecture consisting of two levels, i.e. low and medium voltage nets, is considered, neglecting a representative structure found in real world circumstances. Further abstraction is employed by static profile components, which aggregate intermittent net loads over time, representing high-level abstractions with diminished accuracy. Furthermore, charging and discharging of cars at charging stations utilizes static loads in contrast to variable loads, due to reduction of possible behavior space and computational complexity. In Summary, comparison between our approach and implementation in a simulation framework like SUMO could provide a means for validation of our approach, especially in terms of behavior the transportation system.

In the results obtained from the examples, it has been shown that overall net balance stabilization is higher, when the weight of the power system cost function was higher. In contrast, findings were, when more weight was put on the transportation system cost function, showed lower net balance stabilization. In this context, it remains an issue of validity, whether such an objective distinction between different systems and according objective parameter variation can be applied to real world scenarios, especially when considering a large number of potentially involved objectives. To reduce complexity, several objectives of EVs which are of increased interest to the individual driver such as battery degradation and energy price have been neglected in our model. Therefore, interactions in the context of V2G observed in the examples could prove to be optimistic in contrast to reality. 

\subsection{Novelty}

In contrast to existing approaches, we proposed an approach which allows for holistic modeling of transportation systems and electric networks, allowing for detailed analysis of objectives, constraints and interactions of both systems. Previous approaches have especially shown limitations in scope towards detailed modeling of the electric network and it's individual electric devices and/or infrastructure. In contrast, our approach allows for detailed representation of electric devices and electricity infrastructure. 
In this context, the modeling technique allows to explore and evaluate the interactions between power and transportation systems. 
The modeling technique allows to evaluate the interactions between power and transportation systems through charging stations. For this, a novelty mapping via dynamic channels has been used. Compared to previous work on the modeling technique which employs static channels between components, our approach expanded on this through the addition of dynamic channels mapping connections between different systems dynamically.

Firstly, holistic parametrization of both transportation and power systems allows to consider diverse sets of parameters in different configurations. Secondly, our approach utilizes cost functions in terms of achieving intended system behavior - behavior of both systems can be observed and directly traced to objectives. The objective based approach provides the possibilities for systematically adjusting objectives, i.e. cost functions of individual components resulting in different outcomes, as seen in presented examples. With regard to behavior estimation results, examples showed the feasibility of our approach for modeling V2G interactions. Furthermore, detailed modeling of electric devices enables traceability of energy flows to traffic flows.

