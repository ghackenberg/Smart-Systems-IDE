\title{A parametric approach to holistic transportation system and electric network scenario modeling}

\author{
	\IEEEauthorblockN{Dominik Ascher}
	\IEEEauthorblockA{
		Chair IV: Software \& Systems Engineering\\
		Faculty for Computer Science\\
		Technische Universit\"at M\"unchen\\
		Boltzmannstr. 3, 85748 Garching, Germany\\
		Email: ds.ascher@gmail.com
	}
	\and
	\IEEEauthorblockN{Georg Hackenberg}
	\IEEEauthorblockA{
		Chair IV: Software \& Systems Engineering\\
		Faculty for Computer Science\\
		Technische Universit\"at M\"unchen\\
		Boltzmannstr. 3, 85748 Garching, Germany\\
		Email: hackenbe@in.tum.de
	}
}

\maketitle

\begin{abstract}

Intermittent impacts on the power grid and distribution networks caused by electric vehicles (EVs) remain a major challenge that needs to be addressed to guarantee the sustainability of the widespread adaption of EVs. In this context, efficiently balancing power demand power grids represents a fundamental issue of Vehicle-to-Grid (V2G) applications of EV. In this paper, we investigate an approach for the holistic and granular modeling of transportation systems and electric networks. Specifically, factors at play are electric devices, i.e. producers and consumers within the power system, and individual cars on a traffic network which make up the transportation system. The approach allows for rapid parameter variation and evaluation of individual components of both transportation systems and electric networks in conjunction. Based on examples, we apply our approach to different incremental small-scale scenarios. Obtained simulation results confirm the feasibility of the approach in terms of modeling and parameter variation of transportation system and power systems. Based on simulation results, we conclude with an evaluation of our approach with respect to it's validity and novelty.

\end{abstract}

\begin{keywords}
	
Feasibility study; intelligent transportation systems; smart grids; electric vehicles 

\end{keywords}