\section{Related work}
\label{section:retrospection}

We draw related work mainly from the fields of \textit{modeling and simulation} as well as \textit{control} of transportation and power systems. Hereby we concentrate on integrated approaches for both the transportation and the power system domain.

\textcolor{red}{
In the field of \textit{modeling and simulation} SUMO~\cite{behrisch2011sumo} has seen widespread attention for microscopic simulation of traffic and its straightforward formulation as well as evaluation of large-scale traffic scenarios. In the context of EVs, an extension of SUMO towards modeling electrical and mechanical traction of EVs has been proposed in~\cite{maia2011electric}. Kurcveil et al.~\cite{kurczveil2014implementation} propose an implementation of an energy model and a charging infrastructure in SUMO. The approach allows for modeling representative EV traffic scenarios in SUMO. Soares et al.~\cite{soares2012electric} propose an electric vehicle scenario simulator which enables the definition of electric vehicle scenarios in the context of smart grids and distribution networks. However, its scope is limited to the definition of electric vehicles scenarios and relies on external tools for specific analysis and determining impact on the power grid. Furthermore, the height profile, i.e.\ the elevation slope of routes is not considered in their energetic profile. Increased validity of this approach has been proposed through generating realistic traffic scenarios based on synthetic population generation utilizing a census database \cite{soares2014realistic}.
}

\textcolor{red}{
In contrast, in the field of \textit{control} research has strongly focused on achieving optimal charging decisions for EVs with regard to the power system. Sundstrom et al.~\cite{sundstrom2010planning} consider the capacity of the electricity grid during charging of electric vehicles, thereby avoiding overloading in the grid. Schlote et al.~\cite{schlote2012balanced} implement load-balancing strategies for charging demand at charging points into the routing decisions of electric vehicles. Results showed reduced travel times and congestion within the employed traffic network. Deliami et al.~\cite{deilami2011real} propose a real-time smart load management, which considers random plug-in of plug-in electric vehicles (PEVs) within smart grids, considering scenarios with intermittent charging of PEVs. Alonso et al.~\cite{alonso2014optimal} examine optimal charging scheduling for electric vehicles in smart grids, achieving optimal behavior for charging EVs in a power grid based on a representative low-voltage net topology. The objective for the employed model lies in obtaining daily optimal scheduling of EV demand on transformer substations. Zakariazadeh et al.~\cite{zakariazadeh2014multi} propose a multi-objective scheduling approach for electric vehicles in smart distribution systems, which considers economic and environmental objectives as well as technical constraints of distribution networks.
}

In summary, for representation of a underlying power system, proposed approaches do not model electric devices in detail or only rely on representative electric net topologies or total charging demand. Furthermore, a problem of the presented approaches represents the missing capability to explore holistic scenarios for the transportation and the power system. Considered approaches foremost aim to address the objectives and constraints of the power system, while objectives and constraints of the transportation system are not sufficiently considered. In summary, the presented approaches propose possibilities for parameter variation and modeling of traffic scenarios for EVs. However, for one they microscopically consider the transportation system with it's EVs, but lack possibilities to explore the behavior of the underlying power system or specific electric devices in detail. Consequently, in scenarios, they do not offer possibilities for detailed parameter variation of the power system and it's individual devices. Therefore, holistic transportation scenarios considering the behavior of transportation systems as well as power systems cannot be explored and evaluated.