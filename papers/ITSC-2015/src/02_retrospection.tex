\section{Related work}
\label{section:retrospection}

We draw related work mainly from the fields of \textit{modeling and simulation} as well as \textit{optimal control} of transportation and power systems. Due to space limitations here we concentrate on approaches already integrating both domains.

In the field of \textit{modeling and simulation} SUMO~\cite{behrisch2011sumo} has seen widespread attention for microscopic traffic simulation and its straightforward formulation and evaluation of large-scale traffic scenarios. In the context of EVs, an extension of SUMO with electrical and mechanical traction has been proposed in~\cite{maia2011electric}. Kurcveil et al.~\cite{kurczveil2014implementation} propose an implementation of an energy model and a charging infrastructure in SUMO. The approach allows one to model representative EV traffic scenarios in SUMO. Soares et al.~\cite{soares2012electric} propose an EV scenario simulator which enables the definition of EV scenarios in the context of smart grids and distribution networks. However, its scope is limited to the definition of EV scenarios and relies on external tools for specific analysis and determining impact on the power grid. Furthermore, the height profile, i.e.\ the elevation slope, of routes is not considered in their energetic profile. Increased validity of this approach has been proposed through generating realistic traffic scenarios based on synthetic population generation utilizing a census database \cite{soares2014realistic}.

In the field of \textit{optimal control} one can observe a strong focus on achieving optimal charging decisions for EVs with regard to the power system. Sundstrom et al.~\cite{sundstrom2010planning} consider the capacity of the electric network during charging of EVs, thereby avoiding network overload. Schlote et al.~\cite{schlote2012balanced} implement load-balancing strategies for charging demand at charging points into the routing decisions of EVs. Results showed reduced travel times and congestion within the employed traffic network. Deliami et al.~\cite{deilami2011real} propose a real-time smart load management, which considers random plug-in of plug-in EVs (PEVs) within smart grids, considering scenarios with intermittent charging of PEVs. Alonso et al.~\cite{alonso2014optimal} examine optimal charging scheduling for EVs in smart grids, achieving optimal behavior for charging EVs in an electric network based on a representative low-voltage network topology. Their objective is to obtain daily optimal scheduling of EV demand on transformer substations. Zakariazadeh et al.~\cite{zakariazadeh2014multi} propose a multi-objective scheduling approach for EVs in smart distribution networks, which considers economic and environmental objectives as well as technical constraints of distribution networks.

In summary, we found that existing approaches cover only a limited set of features required to achieve our objective (i.e.\ rapid multi-objective transportation and power system scenario modeling and evaluation). Both modeling/simulation and optimal control models provide sufficient representation of the transportation system while lacking detail about the power system (in particular electric devices). Additionally, the wide variety of transportation and power objectives as well as their relative importance and arising impacts are not addressed sufficiently.